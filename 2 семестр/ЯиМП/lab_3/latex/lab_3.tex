\documentclass[a4paper, 14pt]{extarticle}

% Поля
%--------------------------------------
\usepackage{geometry}
\geometry{a4paper,tmargin=2cm,bmargin=2cm,lmargin=3cm,rmargin=1cm}
%--------------------------------------


%Russian-specific packages
%--------------------------------------
\usepackage[T2A]{fontenc}
\usepackage[utf8]{inputenc}
\usepackage[english, main=russian]{babel}
%--------------------------------------

\usepackage{textcomp}

% Красная строка
%--------------------------------------
\usepackage{indentfirst}
%--------------------------------------


%Graphics
%--------------------------------------
\usepackage{graphicx}
\graphicspath{ {./images/} }
\usepackage{wrapfig}
%--------------------------------------

% Полуторный интервал
%--------------------------------------
\linespread{1.3}
%--------------------------------------

%Выравнивание и переносы
%--------------------------------------
% Избавляемся от переполнений
\sloppy
% Запрещаем разрыв страницы после первой строки абзаца
\clubpenalty=10000
% Запрещаем разрыв страницы после последней строки абзаца
\widowpenalty=10000
%--------------------------------------

%Списки
\usepackage{enumitem}

%Подписи
\usepackage{caption}

%Гиперссылки
\usepackage{hyperref}

\hypersetup {
	unicode=true
}

%Рисунки
%--------------------------------------
\DeclareCaptionLabelSeparator*{emdash}{~--- }
\captionsetup[figure]{labelsep=emdash,font=onehalfspacing,position=bottom}
%--------------------------------------

\usepackage{tempora}

%Листинги
%--------------------------------------
\usepackage{listings}
\lstset{
  basicstyle=\ttfamily\footnotesize,
  %basicstyle=\footnotesize\AnkaCoder,        % the size of the fonts that are used for the code
  breakatwhitespace=false,         % sets if automatic breaks shoulbd only happen at whitespace
  breaklines=true,                 % sets automatic line breaking
  captionpos=t,                    % sets the caption-position to bottom
  inputencoding=utf8,
  frame=single,                    % adds a frame around the code
  keepspaces=true,                 % keeps spaces in text, useful for keeping indentation of code (possibly needs columns=flexible)
  keywordstyle=\bf,       % keyword style
  numbers=left,                    % where to put the line-numbers; possible values are (none, left, right)
  numbersep=5pt,                   % how far the line-numbers are from the code
  xleftmargin=25pt,
  xrightmargin=25pt,
  showspaces=false,                % show spaces everywhere adding particular underscores; it overrides 'showstringspaces'
  showstringspaces=false,          % underline spaces within strings only
  showtabs=false,                  % show tabs within strings adding particular underscores
  stepnumber=1,                    % the step between two line-numbers. If it's 1, each line will be numbered
  tabsize=2,                       % sets default tabsize to 8 spaces
  title=\lstname                   % show the filename of files included with \lstinputlisting; also try caption instead of title
}
%--------------------------------------

%%% Математические пакеты %%%
%--------------------------------------
\usepackage{amsthm,amsfonts,amsmath,amssymb,amscd}  % Математические дополнения от AMS
\usepackage{mathtools}                              % Добавляет окружение multlined
\usepackage[perpage]{footmisc}
%--------------------------------------

%--------------------------------------
%			НАЧАЛО ДОКУМЕНТА
%--------------------------------------

\begin{document}

%--------------------------------------
%			ТИТУЛЬНЫЙ ЛИСТ
%--------------------------------------
\begin{titlepage}
\thispagestyle{empty}
\newpage


%Шапка титульного листа
%--------------------------------------
\vspace*{-60pt}
\hspace{-65pt}
\begin{minipage}{0.3\textwidth}
\hspace*{-20pt}\centering
\includegraphics[width=\textwidth]{emblem}
\end{minipage}
\begin{minipage}{0.67\textwidth}\small \textbf{
\vspace*{-0.7ex}
\hspace*{-6pt}\centerline{Министерство науки и высшего образования Российской Федерации}
\vspace*{-0.7ex}
\centerline{Федеральное государственное бюджетное образовательное учреждение }
\vspace*{-0.7ex}
\centerline{высшего образования}
\vspace*{-0.7ex}
\centerline{<<Московский государственный технический университет}
\vspace*{-0.7ex}
\centerline{имени Н.Э. Баумана}
\vspace*{-0.7ex}
\centerline{(национальный исследовательский университет)>>}
\vspace*{-0.7ex}
\centerline{(МГТУ им. Н.Э. Баумана)}}
\end{minipage}
%--------------------------------------

%Полосы
%--------------------------------------
\vspace{-25pt}
\hspace{-35pt}\rule{\textwidth}{2.3pt}

\vspace*{-20.3pt}
\hspace{-35pt}\rule{\textwidth}{0.4pt}
%--------------------------------------

\vspace{1.5ex}
\hspace{-35pt} \noindent \small ФАКУЛЬТЕТ\hspace{30pt} <<Информатика, искусственный интеллект и системы управления>>

\vspace*{-16pt}
\hspace{47pt}\rule{0.83\textwidth}{0.4pt}

\vspace{0.5ex}
\hspace{-35pt} \noindent \small КАФЕДРА\hspace{50pt} <<Теоретическая информатика и компьютерные технологии>>

\vspace*{-16pt}
\hspace{30pt}\rule{0.866\textwidth}{0.4pt}

\vspace{11em}

\begin{center}
\Large {\bf Лабораторная работа № 3} \\
\large {\bf по курсу <<Языки и методы программирования>>} \\
\large <<Полиморфизм на основе интерфейсов в языке Java>>
\end{center}\normalsize

\vspace{8em}


\begin{flushright}
  {Студент группы ИУ9-21Б Яннаев А. С. \hspace*{15pt}\\
  \vspace{2ex}
  Преподаватель Посевин Д. П.\hspace*{15pt}}
\end{flushright}

\bigskip

\vfill


\begin{center}
\textsl{Москва 2025}
\end{center}
\end{titlepage}
%--------------------------------------
%		КОНЕЦ ТИТУЛЬНОГО ЛИСТА
%--------------------------------------

\renewcommand{\ttdefault}{pcr}

\setlength{\tabcolsep}{3pt}
\newpage
\setcounter{page}{2}

\section{Цель работы}\label{Sect::point}
Приобретение навыков реализации интерфейсов для обеспечения возможности
полиморфной обработки объектов класса.


\section{Задание}\label{Sect::task}

Во время выполнения лабораторной работы требуется разработать на языке Java один из
классов, перечисленных в таблице. В классе должен быть реализован интерфейс Comparable<T>
и переопределён метод toString. В методе main вспомогательного класса Test нужно
продемонстрировать работоспособность разработанного класса путём сортировки массива его
экземпляров.

\begin{tabular}{ l | c }
Вариант №26 & Класс дробей, числитель и знаменатель которых взаимно
\end{tabular}
просты, с естественным порядком на множестве рациональных чисел.



\section{Результаты}\label{Sect::res}

Исходный код программы представлен в ~\ref{lst:code1}, ~\ref{lst:code2}.

\begin{figure}[!htb]
\begin{lstlisting}[language={java},caption={Файл Rational.java},label={lst:code1}]
public class Rational implements Comparable<Rational> {
    int n;
    int d;

    public Rational(int n, int d) {
        if (d == 0) {
            System.out.println("cant be 0");
            System.exit(0); // остановка программы
        }
        int gcd = findGCD(n, d);
        this.n = n / gcd;
        this.d = d / gcd;
        if (this.d < 0) {
            this.n = -this.n;
            this.d = -this.d;
        }
    }

    private int findGCD(int a, int b) {
        while (b != 0) {
            int tmp = b;
            b = a % b;
            a = tmp;
        }
        return a;
    }

    public String toString() { return n + "/" + d; }

    public int compareTo(Rational other) {
        if (this.n * other.d < other.n * this.d) {
            return -1;
        } else if (this.n * other.d > other.n * this.d) {
            return 1;
        } else {
            return 0;
        }
    }
}


\end{lstlisting}
\end{figure}

\begin{figure}[!htb]
\begin{lstlisting}[language={java},caption={Файл Test.java},label={lst:code2}]
import java.util.Arrays;
import java.util.Scanner;

public class Test {
    public static void main(String[] args) {
        Scanner sc = new Scanner(System.in);
        System.out.print("num: ");
        int count = sc.nextInt();

        Rational[] fractions = new Rational[count];

        for (int i = 0; i < count; i++) {
            System.out.print("fraction " + (i + 1) + "; format: 'n d': ");
            int n = sc.nextInt();
            int d = sc.nextInt();

            fractions[i] = new Rational(n, d);
        }

        Arrays.sort(fractions);
        System.out.println("\nsorted: ");
        for (Rational r : fractions) {
            System.out.println(r);
        }
        sc.close();
    }
}
\end{lstlisting}
\end{figure}


Результат запуска представлен на рисунке~\ref{fig:img1}.

\begin{figure}[!htb]
	\centering
	\includegraphics[width=0.6\textwidth]{img1}
\caption{Результат}
\label{fig:img1}
\end{figure}

\end{document}
